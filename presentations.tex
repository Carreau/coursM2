\documentclass[a4paper,10pt]{article}
\usepackage{graphicx}
\usepackage{amsmath}
\usepackage[francais]{babel}
\usepackage[T1]{fontenc}
\newcommand{\bbar}[1]{\underline{\underline{#1}}}
\newcommand{\txt}[1]{\textrm{#1}}
\newcommand{\lam}{\lambda}
\newcommand{\der}[3][]{\frac{\ud^{#1} #2}{\ud #3^{#1}}}
\newcommand{\derp}[3][]{\frac{\partial^{#1} #2}{\partial #3^{#1}}}
\newcommand{\Derp}[3][]{\frac{\textrm{D}^{#1} #2}{\textrm{D} #3^{#1}}}
\newcommand{\ud}{\textrm{d}}
\newcommand{\grad}{\vec{\textrm{grad}}}
\newcommand{\rot}{\vec{\textrm{rot}}}
\renewcommand{\div}{\textrm{div}}
\newenvironment{matrice}{ \left[ \begin{array} }{\end{array} \right]}
\newcommand{\moy}{\left\langle \right\rangle}


\title{}

\section{Leticia Cugliandolo - Physique statistique}

Physique statistique sur les verres de spin, mod�le d'Ising modifi� : $$H = \sum_{ij} - J_{ij} s_i s_j $$ o� les $J_{ij}$ peuvent avoir des signes diff�rents. Appara�t dans des verres physiques. Liens avec des probl�mes d'optimisation d'informatique th�orique, et avec les probl�mes de r�seau de neurones. \'Etude de la dynamique (changements brusques de temp�rature). 

\section{La mati�re molle et l'ESPCI}

2/3 des th�ses en bourses CIFRES. 

\'Elastom�re auto_r�parants, dunes g�antes, mouillage de poudres solubles dans l'eau (quelle est la condition de formation des grumeaux), bulles photo-stimulables (pourquoi les il�ts mangent les cr�pes ? )

\section{Biophysique : m�canique des cellules cill�es}

cf pr�sentation � l'Institut Curie. 

\section{Hydrodynamique et interfaces}

instabilit�s dans un ressaut, instabilit�s entre une goutte et un ressaut, dynamique des nappes flottantes (en lien avec l'industrie du verre flott�), nappes flottantes vibr�es (instabilit�s de forme), reptation et nage du ver c.elegans (fabrication d'un robot nageur). Dynamique rapide des mousses liquides (propagation d'ultrasons dans une mousse).  

